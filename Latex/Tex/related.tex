\section{Related work}\label{sec:related}
We describe the related work from these three aspects,
% integrity verification for hypervisor,
 reconstructed hypervisor, customized hardware, and the same privilege-level isolation.
% The first aspect is considered from the perspective of protecting the hypervisor, and the other three aspects are considered from the perspective of protecting VMs.

\iffalse
%\subsection{Protection for Hypervisor}
\subsection{Integrity Verification for Hypervisor}
In order to ensure the security of the hypervisor during trusted boot and runtime, an effective and commonly used method is to verify the integrity of the hypervisor, and reduce the attack surface. For the security of the hypervisor during trusted boot, paper \cite{Petroni2007Automated} proposes control flow integrity protection policy, by verifying regularly control flow integrity behavior to detect rootkit attacks. However, attacker can detect the regular and bypass the detection. For runtime security of the hypervisor, HyperSafe \cite{Wang2010HyperSafe} and HyperCheck \cite{Wang2010HyperCheck} choose pooling-query method based on SMM to finish integrity verification of hypervisor. However, SMM doesn't support for MMU. And attackers can hide trace during polling-query intervals when comparing to event-driven monitoring.
\fi

%\subsection{Resource Isolation}

\subsection{Customized Hardware }
Some works at hardware level complete the protection of the process by extending the virtualization capabilities or measure the integrity of the Hypervisor\cite{Moon2012Vigilare}\cite{Lee2013KI}\cite{azab2010hypersentry}\cite{rutkowska2008preventing}\cite{Wang2010HyperCheck}\cite{zhang2013hypercheck}\cite{azab2011sice}\cite{szefer2012architectural}. These tasks provide fine-grained isolation of processes and modules from the hardware level. Haven \cite{haven} uses Intel SGX\cite{Hoekstra13cuvillo,Mckeen2013Innovative} to isolate cloud services from other services and prevent cross-domain access. SGX provides fine-grained protection at the application space instead of hypervisor space, and needs developers spend time reconstructing code and dividing code into trusted part or untrusted part. SGX has requirement for version of CPU and is applied on a few platforms.
Datasafe\cite{chen2012software} provides dynammic instantiations of secure data compartments and continuously tracks and propagates hardware tags to identify sentive data by enforcing unbypassable output control.
HyperCoffer\cite{xia2013architecture} is a hardware-software framework to protect the integrity of a guest VM that only trusts the processor chip. HyperCoffer introduced a new mechanism called VM-Shim, which runs in-between a guest VM and the hypervisor.
Some solutions provide protections by using additional hardwares, such as PCI device, TPM, etc.  HyperCheck\cite{Wang2010HyperCheck} \cite{zhang2013hypercheck} leverages SMM and a PCI device to securely generate and transmit the full state of the protected machines to an external server, so that it can provide protection to the integrity of hypervisor. 
% The effort \cite{Cho2016Hardware} combines the advantages of ARM TrustZone and virtualization to improve system performance, and isolate critical process components securely and efficiently.
% H-SVM\cite{Jin2015H} utilizes the hardware extension features of the CPU, and extends SMM microcode to achieve memory resource isolation among virtual machines. It deprives ability of accessing to memory resource by replacing the source code of the original hypervisor to access memory resource.
% Vigilare\cite{Moon2012Vigilare} and KI-Mon \cite{Lee2013KI} provide monitoring for access operations by introducing extra hardware. Vigilare provides a kernel integrity monitor that is architected to snoop the bus traffic of the host system from a separate independent hardware. 
% It adds extra Snooper hardware connections module to the host system for bus snooping. KI-Mon monitors write operation to system bus and handles data to write in order to check rootkit attack.
%However, for cloud providers, these approaches means high-cost and low practicality as long as they are carried out widely.
%KI-Mon monitors write operation on system bus and handles data to write in order to check rootkit attack.

\subsection{Reconstructed Hypervisor }
Except for approaches based on hardware, some works (\cite{shi2017deconstructing}\cite{shi2017deconstructing}\cite{hyperlock}) pay attention to software isolation. Pre-allocating physical resource and completed isolated environment for every VM can avoid VM cross-domain attack, and data leakage attack. NOVA\cite{shi2017deconstructing} divides hypervisor into micro-hypervisor and user hypervisor running in root mode, adopts an idea which is similar to fault domain isolation to guarantee an isolated user hypervisor for every VM. The drawback of this approach is the lack of fractional traditional hypervisor functions. HyperLock \cite{hyperlock} prepares backup KVM for every VM by copying KVM code, and ensures every VM run in own isolated space. 
Nexen\cite{shi2017deconstructing} reconstructs the XEN hypervisor into one privileged security monitor, one component for shared service, backup XEN code and data for every VM, to resist attacker from exploiting known XEN vulnerabilities.
Some solutions\cite{ben2010turtles,strackx2012fides} propose to insert another layer below the hypervisor to measure and guarantee the integrity of the hypervisor. GuardHype\cite{carbone2008taming} mediates the access of third-party hypervisors to the hardware virtualization extensions, effectively acting as a hypervisor for hypervisors. CloudAuditor \cite{Wang2016CloudAuditorAC}, Nosv \cite{REN2017137}, CloudViosr-D \cite{mi2020mostly}, and the Turtles Project \cite{183330}, reconstruct the Hypervisor by introducing a higher privilege level than the origin hypervisor. 

These approaches redesign hypervisor greatly. In contrary, HyperPS adopts a feasible way to isolate VM without lots of modification of hypervisor. 

\subsection{The Same Privilege Level Isolation}
Some efforts, 
ED-Monitor\cite{Deng2017Dancing}, SKEE\cite{azab2016skee} and SecPod\cite{Wang2015SecPod},\cite{Deng2017Dancing}, adopt the same privilege-level idea to avoid performance overhead of inter-level translation.
ED-Monitor presents a novel approach that enables practical event-driven monitoring for compromised hypervisor in cloud computing, adopts "the same privilege level" protection against an untrusted hypervisor. The created monitor is placed at the same privilege level and in the same space with hypervisor. It relies on the mutual-protection of a unique pair of the techniques: Instrumentation-based Privilege Restriction (IPR) and Address Space Randomization (ASR). At the high level, IPR intercepts the most privileged operations in the Hypervisor and transfers these operations to ED-monitor, while ASR hides ED-monitor in the address space from the Hypervisor.
 %SKEE provides a lightweight secure kernel-level execution environment, it is placed at the same privilege-level with kernel. SKEE is exclusively designed for commodity ARM platforms using system characteristics of ARM. 
SKEE can only be applied to limited levels of system software in comparison with HyperPS. First, targeting ARM's 32-bit architecture, SKEE capitalizes mainly on Translation Table Base Control Register (TTBCR) for dynamic page table activation. To be more specific, SKEE creates separate page tables for the secure world and activates it in a timely manner by modifying the N field of TTBCR. However, as this hardware feature is only defined in the kernel privilege level on AArch32, SKEE is not commonly applicable to different levels of system software, such as hypervisors. %Unfortunately, SKEE faces a similar limitation on ARM's 64-bit architecture. SKEE is more focused on using the features of the ARM platform while HyperPS has no dependence on multi-platforms.
%%%%%%%The difference between HyperPS and SKEE is that HyperPS uses two sets of page tables to create the execution environment, and SKEE uses one set of page table. The design of the switch gate for HyperPS and SKEE is also different. SKEE is more focused on using the characteristic of the ARM platform while HyperPS has no dependence on multi-platforms.
%When kernel is compromised, an attacker cannot break the isolation between SKEE and the kernel, and the security of internal security tools placed at secure isolated environment is guaranteed.
SecPod, an extensible approach for virtualization-based security systems that can provide both strong isolation and the compatibility with modern hardware. The biggest difference between SecPod and HyperPS is that SecPod creates the secure isolation environment for every VM. SecPod solves the problem of VM address mapping with the assistance of shadow page table (SPT) technology.
% SecPod has two key techniques: paging delegation delegates and audits the kernel's paging operations to a secure space; execution trapping intercepts the (compromised) kernel's attempts to subvert SecPod by misusing privileged instructions.

We neither adopt software at a higher level than the hypervisor, nor use customized hardware. Inspired by the same privilege-level, we propose HyperPS Space placed at the same privilege-level with hypervisor. HyperPS is independent on multi-platforms and practical for cloud providers.

