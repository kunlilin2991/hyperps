\section{Background and Threat Model}%
\label{sec:background_and_threat_model}

% 这个章节也是由初稿中的Paranoma中分离出来,但是原来的全景图更换为威胁模型图,要重新写这部分内容
This section presents some necessary background knowledge to facilitate readers to better understand HyperPS and threat model. 

% depicted in Figure \ref{pic:panorama}.


\subsubsection{QEMU-KVM}%
\label{ssub:qemu_kvm}
QEMU is a generic and open source machine emulator and virtualizer. QEMU can use other hypervisor like \verb|Xen| or KVM to use CPU extensions for virtualization. When used as a virtualizer, QEMU achieves near native performances by executing the guest code directly on the host CPU.
Kernel-based Virtual Machine (KVM) is an open source virtualization technology that converts Linux into a type-1 (bare-metal) hypervisor. KVM is a part of the Linux kernel that shares all the linux kernel's operating system-level components -such as the memory manager, process scheduler, security manager, and more to run VMs. Every VM is implemented as a regular Linux process, scheduled by the standard Linux scheduler, with dedicated virtual hardware like a network card, memory, and disks. KVM mainly consists of a loadable kernel module, kvm.ko, that provides the core virtualization infrastructure and a processor specific module, kvm-intel.ko or kvm-amd.ko.
In virtualization environment, KVM does not work on its own. It is only an API provided by the kernel for userspace. End users typically use KVM through QEMU where it is present as an acceleration method.
For the QEMU-KVM architecture, KVM interacts with QEMU (QEMU runs at the user space) in two ways: through device file \verb|/dev/kvm| and through memory mapped pages
% KVM interacts with user space - in most case QEMU - in two ways: through device file \verb|/dev/kvm| and through memory mapped pages.
Memory mapped pages are used for bulk transfer of data between QEMU and KVM. \verb|/dev/kvm| is the main API exposed by KVM. It supports a set of \verb|ioctl|s which allow QEMU to manage VMs and interact with them.
% 怎么引出HostOS的问题
% Since KVM is a part of the linux kernel, the Linux and the QEMU
% leverages the linux kernel's system-level



\subsubsection{VMCS}%
\label{ssub:vmcs}
Virtual-machine Control Structure is a data structure used in Virtual Machine eXtensions (VMX). It controls VMX non-root operations (guest virtual machine execution operations) and VMX transitions. Access to the VMCS is managed through the VMCS pointer (one per logical processor). The VMCS pointer is read and written suing the instructions \verb|VMPTRST| and \verb|VMPTRLD|. In general, the hypervisor configures a VMCS using \verb|VMREAD|, \verb|VMWRITE|, and \verb|VMCLEAR| instructions. A hypervisor could use a different VMCS for each virtual machine that it supports. For a virtual machine with multiple logical processor, the hypervisor could use a different for each logical processor. A logical processor may also maintain a number of VMCSs that are active, however, at any given time, at most one of the active VMCSs is the \textbf{current} VMCS. VMX instructions operate only on the \textbf{current} VMCS. 

%这段写的不好,没有写出危害的具体内容,应该突出VMCS就是唯一的接口,VM的所有的行为都是在VMCS的instruct下的
A compromised HostOS/hypervisor can force the guest virtual machine exit by tramper VM-Execution Control fields and VM-exit control fields in VMCS and tramper the guest virtual machine by writing Guest-State Area fields. 

% 一个被危害的hypervisor可以强迫虚拟机退出,并通过篡改其vmcs中的field等信息实现对guest VM 的危害。

% \subsection{Second Level Address Translation}%
% \label{sub:second_level_address_translation}
% Second Level Address Translation (SLAT) is a hardware-assisted virtualization technology which makes it possible to quick manage the physical memory without lots of VM exits. Extended Page Table (EPT) is the intel's implementation of SLAT, while ARM names its implementation as Stage-2 Page-tables.


\subsubsection{EPT}%
\label{ssub:ept}
Intel implements it Second Level Address Translation (SLAT) as Extended Page Table (EPT). 
EPT allows each virtual machine to manage its page table (not the EPT), without giving access to the underlying host machine's MMU Hardware. Thus, EPT reduces the need for hypervisor to keep syncing the shadow pages eliminating the overhead.
If EPT is enabled, guest-physical addresses are translated by traversing a set of EPT paging structures to produce physical addresses that are used to access memory.
In addition to translating a guest-physical address to a physical address, EPT specifies the privileges that software is allowed when accessing the address. Attempts at disallowed accesses are called EPT violations and cause VM exits.

A compromised HostOS/hypervisor could tramper EPT paging structures so that the virtual machine will execute arbitrary malicious code without knowing it. 
Moreover, a compromised HostOS/Hypervisor could access any data in the guest virtual machine with the help of virtual machine introspection.




%\subsection{Attacks and Threat Model} \label{sub:thretmodel}
\subsection{Attack Surface}\label{sub:attacksurface}
\iffalse
在这一章节中,我们首先给出攻击者的攻击路径,这个攻击路径并不需要单独画图,因为上面的图中已经阐释清楚了,攻击者有哪些方式可以危害虚拟机,只需要在图中更加明确的标识出来就可以了。然后再给出本文的威胁模型。

首先一句话尽可能简洁的描述出攻击者的攻击目的,

在这一章节中,我们首先描述在云环境中,攻击者如何攻击利用HostOS的漏洞实现对客户虚拟机的攻击。
其次,我们给出敌手的能力。
\fi

In this section, we first present how attackers tamper virtual machines through exploiting vulnerabilities in HostOS/Hypervisor. Then, we present several typical attack models/examples targeting at VMCS and EPT. At last, we explain some additional attacks (not attacks relative to virtualization components)
% We present attack surface of hypervisor and

\begin{figure}[htpb]
    \centering
    \includegraphics[width=1\linewidth]{IMG/threat.pdf}
    \caption{The Attack Paths in the QEMU-KVM Cloud Environment}%
    \label{fig:threat}
\end{figure}

% depicted in Figure \ref{pic:panorama}.
%TODO 将图中的具体的攻击名称 1 2 3 4 等添加进去。
%At last, we illustrate the adversary's abilities. 

\subsubsection{How attacker subvert VM}
As illustrated in Section \ref{sub:background}, Hypervisor controls the execution of VMs through VMCSs, and manages VMs' physical memory through EPT. As such, these two data structures become the key target for adversary to tamper VMs. 
As depicted in Figure \ref{fig:threat}, 
An adversary can exploit vulnerabilities to ``jail-break'' into the HostOS/Hypervisor, while he can also subvert HostOS/Hypervisor by exploiting vulnerabilities in the HostOS kernel.
In this paper, we hypothesize the HostOS/Hypervisor has already been compromised, we attempt to protect VMs under the compromised HostOS/Hypervisor. 
After compromising the HostOS/Hypervisor, the attacker will inject malicious code to subvert VMs by tampering VMCSs and EPTs.
In details, we assume that the attacker can tamper VMCSs and EPTs with one of the following methods. First, the attacker can tamper any field in these two data structures with \verb|VMX| instructions, if the attacker has gained the root processor privilege. Second, the attacker can tamper these two data structures through direct memory write operations. The attacker can write these two data structures either through Direct Memory Access (DMA) or through regular memory access, if the attacker acquired these two data structures' memory locations before.  
%direct write to fields in these two data structures or 
%TODO 在这里要修改图,图中要添加内核的其他部分的功能,去掉内部管理者,保留虚拟机逃逸部分,同时对这部分的内容做虚线处理。

%这里,我们给出几个攻击模型,举例说明攻击者是如何危害上面的虚拟机的。

\subsubsection{Attack Examples}
We present several attacks to illustrate how an attack subvert VMs by through tampering VMCS and EPT.

\paragraph{VMCS Subversion}
Attackers can tamper fields of VMCS in VM Exit stage. For example, modifying the value of \verb|HOST_RIP| register and writing a malicious program address to this register will cause a control flow hijacking attack. Modifying the value of privilege register, CR0, will close the DEP mechanism, and modifying the value of CR4 register will close the SMEP mechanism.


\paragraph{EPT Subversion}
Malicious modification of EPT will break the isolation between virtual machines. A vicious VM can break the isolation between VMs by tampering with EPT entries, and access the memory of the victim VM at his vicious will. For example, the attacker can conduct remapping attack and double mapping attack to inspect a victim VM. 

For the double mapping attack, the attacker first controls and compromises a VM, then obtains the privilege of the hypervisor through the virtual machine escape attack, and maliciously accesses the VMCS structure to obtain the value of \verb|EPTP|. The attack process is as shown in Figure \ref{pic:remap}. 

In this way, the EPT address of the attacker virtual machine, VM1, and the victim virtual machine, VM2, are respectively obtained. And for a guest virtual address in VM2, A, the corresponding real physical address is B. For VM1, the real physical address corresponding to the guest virtual address C is D, then D is modified to be B by modifying the value of the last page item of EPT. Then VM1 can access the data of VM2 successfully, this process is called address double mapping.

For the remapping attack, there are VM1 (attacker) and VM2 (victim). A physical page (A) used by VM2 is released after being used. After A is released, VM1 remaps to A. So that the guest virtual address of VM1 points to the physical page A. By this way, VM1 can access the information on A used by VM2, causing information leakage.


\begin{figure}
    \centering
    \includegraphics[width=0.8\linewidth]{IMG/remap.pdf}
    \caption{Diagram of Remapping Attack}
    \label{pic:remap}
\end{figure}

\subsubsection{Attacks to Kernel Page Tables}
We also take into account the fact that attacker already knows the deployment of HyperPS. As depicted in Figure \ref{fig:threat}, key components, such as Kernel Page Tables, Control Registers, to create the secure and isolated execution environment are also protected by us. More details about the creation of the secure and isolated execution environment are illustrated in Section. 
%TODO 添加关于章节的引用
 


\subsection{Threat Model} \label{sub:threatmodel}
In this paper, 
We assume that the HostOS/Hypervisor has been compromised and controlled by the powerful adversary. The adversary can turn off kernel security mechanisms, such as DEP, SMEP, SMAP, and so on. The adversary can tamper fields of VMCSs and EPTs with dedicated malicious values. The adversary can also tamper VMCSs and EPTs through DMA write or regular memory access to them.  

We assume that the adversary does not possess the capability to conduct side channel attack and Hardware-based attack. We also assume that the adversary is unwilling to conduct the Denial of Service attack (DOS). In this paper, we assume that hardware resources are trusted, including processor, buses, BIOS/UEFI, and so on. 





















