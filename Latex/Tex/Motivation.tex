\section{Motivation}%
\label{sec:motivation}
% 这部分内容由原本的Panorama中分离出来,为单独的章节。
Among the many services that cloud service providers need to provide to cloud tenants, ensuring the integrity and security of virtual machines is the key factor in gaining user trust. For each VM, the cloud service provider needs to ensure that the user's business in the VM will not be maliciously tampered with, and the data will not be stolen. For different VMs, cloud service providers need to ensure effective isolation between different VMs. However, in the QEMU-KVM architecture, Linux as HostOS is not only a hypervisor that is responsible for providing virtualization services, but also a normal operating system with full functions. In a real business deployment, even if cloud service providers adopt customized Linux as the HostOS, the Linux kernel still contains a large number of various subsystems that have nothing to do with virtualization. These subsystems, especially kernel drivers with complex sources, inevitably contain countless exploitable vulnerabilities. These subsystems provide the attacker with a huge attack surface. Even if the attacker does not find any exploitable QEMU or KVM vulnerabilities, he can still tamper with the virtualization component by exploiting the vulnerabilities of other Linux subsystems. Therefore, cloud service providers need a further think about how to ensure the security of VMs under the premise that the HostOS/Hypervisor has been compromised. This is the motivation of our HyperPS.

