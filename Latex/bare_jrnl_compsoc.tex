
%% bare_jrnl_compsoc.tex
%% V1.4b
%% 2015/08/26
%% by Michael Shell
%% See:
%% http://www.michaelshell.org/
%% for current contact information.
%%
%% This is a skeleton file demonstrating the use of IEEEtran.cls
%% (requires IEEEtran.cls version 1.8b or later) with an IEEE
%% Computer Society journal paper.
%%
%% Support sites:
%% http://www.michaelshell.org/tex/ieeetran/
%% http://www.ctan.org/pkg/ieeetran
%% and
%% http://www.ieee.org/

%%*************************************************************************
%% Legal Notice:
%% This code is offered as-is without any warranty either expressed or
%% implied; without even the implied warranty of MERCHANTABILITY or
%% FITNESS FOR A PARTICULAR PURPOSE! 
%% User assumes all risk.
%% In no event shall the IEEE or any contributor to this code be liable for
%% any damages or losses, including, but not limited to, incidental,
%% consequential, or any other damages, resulting from the use or misuse
%% of any information contained here.
%%
%% All comments are the opinions of their respective authors and are not
%% necessarily endorsed by the IEEE.
%%
%% This work is distributed under the LaTeX Project Public License (LPPL)
%% ( http://www.latex-project.org/ ) version 1.3, and may be freely used,
%% distributed and modified. A copy of the LPPL, version 1.3, is included
%% in the base LaTeX documentation of all distributions of LaTeX released
%% 2003/12/01 or later.
%% Retain all contribution notices and credits.
%% ** Modified files should be clearly indicated as such, including  **
%% ** renaming them and changing author support contact information. **
%%*************************************************************************


% *** Authors should verify (and, if needed, correct) their LaTeX system  ***
% *** with the testflow diagnostic prior to trusting their LaTeX platform ***
% *** with production work. The IEEE's font choices and paper sizes can   ***
% *** trigger bugs that do not appear when using other class files.       ***                          ***
% The testflow support page is at:
% http://www.michaelshell.org/tex/testflow/


\documentclass[10pt,journal,compsoc]{IEEEtran}
%
% If IEEEtran.cls has not been installed into the LaTeX system files,
% manually specify the path to it like:
% \documentclass[10pt,journal,compsoc]{../sty/IEEEtran}





% Some very useful LaTeX packages include:
% (uncomment the ones you want to load)


% *** MISC UTILITY PACKAGES ***
%
%\usepackage{ifpdf}
% Heiko Oberdiek's ifpdf.sty is very useful if you need conditional
% compilation based on whether the output is pdf or dvi.
% usage:
% \ifpdf
%   % pdf code
% \else
%   % dvi code
% \fi
% The latest version of ifpdf.sty can be obtained from:
% http://www.ctan.org/pkg/ifpdf
% Also, note that IEEEtran.cls V1.7 and later provides a builtin
% \ifCLASSINFOpdf conditional that works the same way.
% When switching from latex to pdflatex and vice-versa, the compiler may
% have to be run twice to clear warning/error messages.






% *** CITATION PACKAGES ***
%
\ifCLASSOPTIONcompsoc
  % IEEE Computer Society needs nocompress option
  % requires cite.sty v4.0 or later (November 2003)
  \usepackage[nocompress]{cite}
\else
  % normal IEEE
  \usepackage{cite}
\fi
% cite.sty was written by Donald Arseneau
% V1.6 and later of IEEEtran pre-defines the format of the cite.sty package
% \cite{} output to follow that of the IEEE. Loading the cite package will
% result in citation numbers being automatically sorted and properly
% "compressed/ranged". e.g., [1], [9], [2], [7], [5], [6] without using
% cite.sty will become [1], [2], [5]--[7], [9] using cite.sty. cite.sty's
% \cite will automatically add leading space, if needed. Use cite.sty's
% noadjust option (cite.sty V3.8 and later) if you want to turn this off
% such as if a citation ever needs to be enclosed in parenthesis.
% cite.sty is already installed on most LaTeX systems. Be sure and use
% version 5.0 (2009-03-20) and later if using hyperref.sty.
% The latest version can be obtained at:
% http://www.ctan.org/pkg/cite
% The documentation is contained in the cite.sty file itself.
%
% Note that some packages require special options to format as the Computer
% Society requires. In particular, Computer Society  papers do not use
% compressed citation ranges as is done in typical IEEE papers
% (e.g., [1]-[4]). Instead, they list every citation separately in order
% (e.g., [1], [2], [3], [4]). To get the latter we need to load the cite
% package with the nocompress option which is supported by cite.sty v4.0
% and later. Note also the use of a CLASSOPTION conditional provided by
% IEEEtran.cls V1.7 and later.





% *** GRAPHICS RELATED PACKAGES ***
%
\ifCLASSINFOpdf
  % \usepackage[pdftex]{graphicx}
  % declare the path(s) where your graphic files are
  % \graphicspath{{../pdf/}{../jpeg/}}
  % and their extensions so you won't have to specify these with
  % every instance of \includegraphics
  % \DeclareGraphicsExtensions{.pdf,.jpeg,.png}
\else
  % or other class option (dvipsone, dvipdf, if not using dvips). graphicx
  % will default to the driver specified in the system graphics.cfg if no
  % driver is specified.
  % \usepackage[dvips]{graphicx}
  % declare the path(s) where your graphic files are
  % \graphicspath{{../eps/}}
  % and their extensions so you won't have to specify these with
  % every instance of \includegraphics
  % \DeclareGraphicsExtensions{.eps}
\fi
% graphicx was written by David Carlisle and Sebastian Rahtz. It is
% required if you want graphics, photos, etc. graphicx.sty is already
% installed on most LaTeX systems. The latest version and documentation
% can be obtained at: 
% http://www.ctan.org/pkg/graphicx
% Another good source of documentation is "Using Imported Graphics in
% LaTeX2e" by Keith Reckdahl which can be found at:
% http://www.ctan.org/pkg/epslatex
%
% latex, and pdflatex in dvi mode, support graphics in encapsulated
% postscript (.eps) format. pdflatex in pdf mode supports graphics
% in .pdf, .jpeg, .png and .mps (metapost) formats. Users should ensure
% that all non-photo figures use a vector format (.eps, .pdf, .mps) and
% not a bitmapped formats (.jpeg, .png). The IEEE frowns on bitmapped formats
% which can result in "jaggedy"/blurry rendering of lines and letters as
% well as large increases in file sizes.
%
% You can find documentation about the pdfTeX application at:
% http://www.tug.org/applications/pdftex






% *** MATH PACKAGES ***
%
%\usepackage{amsmath}
% A popular package from the American Mathematical Society that provides
% many useful and powerful commands for dealing with mathematics.
%
% Note that the amsmath package sets \interdisplaylinepenalty to 10000
% thus preventing page breaks from occurring within multiline equations. Use:
%\interdisplaylinepenalty=2500
% after loading amsmath to restore such page breaks as IEEEtran.cls normally
% does. amsmath.sty is already installed on most LaTeX systems. The latest
% version and documentation can be obtained at:
% http://www.ctan.org/pkg/amsmath





% *** SPECIALIZED LIST PACKAGES ***
%
%\usepackage{algorithmic}
% algorithmic.sty was written by Peter Williams and Rogerio Brito.
% This package provides an algorithmic environment fo describing algorithms.
% You can use the algorithmic environment in-text or within a figure
% environment to provide for a floating algorithm. Do NOT use the algorithm
% floating environment provided by algorithm.sty (by the same authors) or
% algorithm2e.sty (by Christophe Fiorio) as the IEEE does not use dedicated
% algorithm float types and packages that provide these will not provide
% correct IEEE style captions. The latest version and documentation of
% algorithmic.sty can be obtained at:
% http://www.ctan.org/pkg/algorithms
% Also of interest may be the (relatively newer and more customizable)
% algorithmicx.sty package by Szasz Janos:
% http://www.ctan.org/pkg/algorithmicx




% *** ALIGNMENT PACKAGES ***
%
%\usepackage{array}
% Frank Mittelbach's and David Carlisle's array.sty patches and improves
% the standard LaTeX2e array and tabular environments to provide better
% appearance and additional user controls. As the default LaTeX2e table
% generation code is lacking to the point of almost being broken with
% respect to the quality of the end results, all users are strongly
% advised to use an enhanced (at the very least that provided by array.sty)
% set of table tools. array.sty is already installed on most systems. The
% latest version and documentation can be obtained at:
% http://www.ctan.org/pkg/array


% IEEEtran contains the IEEEeqnarray family of commands that can be used to
% generate multiline equations as well as matrices, tables, etc., of high
% quality.




% *** SUBFIGURE PACKAGES ***
%\ifCLASSOPTIONcompsoc
%  \usepackage[caption=false,font=footnotesize,labelfont=sf,textfont=sf]{subfig}
%\else
%  \usepackage[caption=false,font=footnotesize]{subfig}
%\fi
% subfig.sty, written by Steven Douglas Cochran, is the modern replacement
% for subfigure.sty, the latter of which is no longer maintained and is
% incompatible with some LaTeX packages including fixltx2e. However,
% subfig.sty requires and automatically loads Axel Sommerfeldt's caption.sty
% which will override IEEEtran.cls' handling of captions and this will result
% in non-IEEE style figure/table captions. To prevent this problem, be sure
% and invoke subfig.sty's "caption=false" package option (available since
% subfig.sty version 1.3, 2005/06/28) as this is will preserve IEEEtran.cls
% handling of captions.
% Note that the Computer Society format requires a sans serif font rather
% than the serif font used in traditional IEEE formatting and thus the need
% to invoke different subfig.sty package options depending on whether
% compsoc mode has been enabled.
%
% The latest version and documentation of subfig.sty can be obtained at:
% http://www.ctan.org/pkg/subfig




% *** FLOAT PACKAGES ***
%
%\usepackage{fixltx2e}
% fixltx2e, the successor to the earlier fix2col.sty, was written by
% Frank Mittelbach and David Carlisle. This package corrects a few problems
% in the LaTeX2e kernel, the most notable of which is that in current
% LaTeX2e releases, the ordering of single and double column floats is not
% guaranteed to be preserved. Thus, an unpatched LaTeX2e can allow a
% single column figure to be placed prior to an earlier double column
% figure.
% Be aware that LaTeX2e kernels dated 2015 and later have fixltx2e.sty's
% corrections already built into the system in which case a warning will
% be issued if an attempt is made to load fixltx2e.sty as it is no longer
% needed.
% The latest version and documentation can be found at:
% http://www.ctan.org/pkg/fixltx2e


%\usepackage{stfloats}
% stfloats.sty was written by Sigitas Tolusis. This package gives LaTeX2e
% the ability to do double column floats at the bottom of the page as well
% as the top. (e.g., "\begin{figure*}[!b]" is not normally possible in
% LaTeX2e). It also provides a command:
%\fnbelowfloat
% to enable the placement of footnotes below bottom floats (the standard
% LaTeX2e kernel puts them above bottom floats). This is an invasive package
% which rewrites many portions of the LaTeX2e float routines. It may not work
% with other packages that modify the LaTeX2e float routines. The latest
% version and documentation can be obtained at:
% http://www.ctan.org/pkg/stfloats
% Do not use the stfloats baselinefloat ability as the IEEE does not allow
% \baselineskip to stretch. Authors submitting work to the IEEE should note
% that the IEEE rarely uses double column equations and that authors should try
% to avoid such use. Do not be tempted to use the cuted.sty or midfloat.sty
% packages (also by Sigitas Tolusis) as the IEEE does not format its papers in
% such ways.
% Do not attempt to use stfloats with fixltx2e as they are incompatible.
% Instead, use Morten Hogholm'a dblfloatfix which combines the features
% of both fixltx2e and stfloats:
%
% \usepackage{dblfloatfix}
% The latest version can be found at:
% http://www.ctan.org/pkg/dblfloatfix




%\ifCLASSOPTIONcaptionsoff
%  \usepackage[nomarkers]{endfloat}
% \let\MYoriglatexcaption\caption
% \renewcommand{\caption}[2][\relax]{\MYoriglatexcaption[#2]{#2}}
%\fi
% endfloat.sty was written by James Darrell McCauley, Jeff Goldberg and 
% Axel Sommerfeldt. This package may be useful when used in conjunction with 
% IEEEtran.cls'  captionsoff option. Some IEEE journals/societies require that
% submissions have lists of figures/tables at the end of the paper and that
% figures/tables without any captions are placed on a page by themselves at
% the end of the document. If needed, the draftcls IEEEtran class option or
% \CLASSINPUTbaselinestretch interface can be used to increase the line
% spacing as well. Be sure and use the nomarkers option of endfloat to
% prevent endfloat from "marking" where the figures would have been placed
% in the text. The two hack lines of code above are a slight modification of
% that suggested by in the endfloat docs (section 8.4.1) to ensure that
% the full captions always appear in the list of figures/tables - even if
% the user used the short optional argument of \caption[]{}.
% IEEE papers do not typically make use of \caption[]'s optional argument,
% so this should not be an issue. A similar trick can be used to disable
% captions of packages such as subfig.sty that lack options to turn off
% the subcaptions:
% For subfig.sty:
% \let\MYorigsubfloat\subfloat
% \renewcommand{\subfloat}[2][\relax]{\MYorigsubfloat[]{#2}}
% However, the above trick will not work if both optional arguments of
% the \subfloat command are used. Furthermore, there needs to be a
% description of each subfigure *somewhere* and endfloat does not add
% subfigure captions to its list of figures. Thus, the best approach is to
% avoid the use of subfigure captions (many IEEE journals avoid them anyway)
% and instead reference/explain all the subfigures within the main caption.
% The latest version of endfloat.sty and its documentation can obtained at:
% http://www.ctan.org/pkg/endfloat
%
% The IEEEtran \ifCLASSOPTIONcaptionsoff conditional can also be used
% later in the document, say, to conditionally put the References on a 
% page by themselves.




% *** PDF, URL AND HYPERLINK PACKAGES ***
%
%\usepackage{url}
% url.sty was written by Donald Arseneau. It provides better support for
% handling and breaking URLs. url.sty is already installed on most LaTeX
% systems. The latest version and documentation can be obtained at:
% http://www.ctan.org/pkg/url
% Basically, \url{my_url_here}.





% *** Do not adjust lengths that control margins, column widths, etc. ***
% *** Do not use packages that alter fonts (such as pslatex).         ***
% There should be no need to do such things with IEEEtran.cls V1.6 and later.
% (Unless specifically asked to do so by the journal or conference you plan
% to submit to, of course. )


% correct bad hyphenation here
\hyphenation{op-tical net-works semi-conduc-tor}


\begin{document}
%
% paper title
% Titles are generally capitalized except for words such as a, an, and, as,
% at, but, by, for, in, nor, of, on, or, the, to and up, which are usually
% not capitalized unless they are the first or last word of the title.
% Linebreaks \\ can be used within to get better formatting as desired.
% Do not put math or special symbols in the title.
% \title{Bare Demo of IEEEtran.cls for\\ IEEE Computer Society Journals}
\title{HyperPS: A Hypervisor Monitoring Approach Based on Privilege Separation}
%
%
% author names and IEEE memberships
% note positions of commas and nonbreaking spaces ( ~ ) LaTeX will not break
% a structure at a ~ so this keeps an author's name from being broken across
% two lines.
% use \thanks{} to gain access to the first footnote area
% a separate \thanks must be used for each paragraph as LaTeX2e's \thanks
% was not built to handle multiple paragraphs
%
%
%\IEEEcompsocitemizethanks is a special \thanks that produces the bulleted
% lists the Computer Society journals use for "first footnote" author
% affiliations. Use \IEEEcompsocthanksitem which works much like \item
% for each affiliation group. When not in compsoc mode,
% \IEEEcompsocitemizethanks becomes like \thanks and
% \IEEEcompsocthanksitem becomes a line break with idention. This
% facilitates dual compilation, although admittedly the differences in the
% desired content of \author between the different types of papers makes a
% one-size-fits-all approach a daunting prospect. For instance, compsoc 
% journal papers have the author affiliations above the "Manuscript
% received ..."  text while in non-compsoc journals this is reversed. Sigh.

\author{Kunli~Lin,
        Kun~Zhang,
        and~Bibo~Tu,% <-this % stops a space
\IEEEcompsocitemizethanks{\IEEEcompsocthanksitem Bibo Tu was with Institute of Information Engineering,
Chinese Academy of Sciences, Beijing, China \protect\\
% note need leading \protect in front of \\ to get a newline within \thanks as
% \\ is fragile and will error, could use \hfil\break instead.
E-mail:tubibo@iie.ac.cn 
\IEEEcompsocthanksitem Kunli Lin and Bibo Tu are with Chinese Academy of Sciences.}}% <-this % stops an unwanted space
% \thanks{Manuscript received April 19, 2005; revised August 26, 2015.}}

% note the % following the last \IEEEmembership and also \thanks - 
% these prevent an unwanted space from occurring between the last author name
% and the end of the author line. i.e., if you had this:
% 
% \author{....lastname \thanks{...} \thanks{...} }
%                     ^------------^------------^----Do not want these spaces!
%
% a space would be appended to the last name and could cause every name on that
% line to be shifted left slightly. This is one of those "LaTeX things". For
% instance, "\textbf{A} \textbf{B}" will typeset as "A B" not "AB". To get
% "AB" then you have to do: "\textbf{A}\textbf{B}"
% \thanks is no different in this regard, so shield the last } of each \thanks
% that ends a line with a % and do not let a space in before the next \thanks.
% Spaces after \IEEEmembership other than the last one are OK (and needed) as
% you are supposed to have spaces between the names. For what it is worth,
% this is a minor point as most people would not even notice if the said evil
% space somehow managed to creep in.



% The paper headers
\markboth{Journal of \LaTeX\ Class Files,~Vol.~14, No.~8, August~2015}%
{Shell \MakeLowercase{\textit{et al.}}: Bare Demo of IEEEtran.cls for Computer Society Journals}
% The only time the second header will appear is for the odd numbered pages
% after the title page when using the twoside option.
% 
% *** Note that you probably will NOT want to include the author's ***
% *** name in the headers of peer review papers.                   ***
% You can use \ifCLASSOPTIONpeerreview for conditional compilation here if
% you desire.



% The publisher's ID mark at the bottom of the page is less important with
% Computer Society journal papers as those publications place the marks
% outside of the main text columns and, therefore, unlike regular IEEE
% journals, the available text space is not reduced by their presence.
% If you want to put a publisher's ID mark on the page you can do it like
% this:
%\IEEEpubid{0000--0000/00\$00.00~\copyright~2015 IEEE}
% or like this to get the Computer Society new two part style.
%\IEEEpubid{\makebox[\columnwidth]{\hfill 0000--0000/00/\$00.00~\copyright~2015 IEEE}%
%\hspace{\columnsep}\makebox[\columnwidth]{Published by the IEEE Computer Society\hfill}}
% Remember, if you use this you must call \IEEEpubidadjcol in the second
% column for its text to clear the IEEEpubid mark (Computer Society jorunal
% papers don't need this extra clearance.)



% use for special paper notices
%\IEEEspecialpapernotice{(Invited Paper)}



% for Computer Society papers, we must declare the abstract and index terms
% PRIOR to the title within the \IEEEtitleabstractindextext IEEEtran
% command as these need to go into the title area created by \maketitle.
% As a general rule, do not put math, special symbols or citations
% in the abstract or keywords.
\IEEEtitleabstractindextext{%
\begin{abstract}
The abstract goes here.
\end{abstract}

% Note that keywords are not normally used for peerreview papers.
\begin{IEEEkeywords}
Computer Society, IEEE, IEEEtran, journal, \LaTeX, paper, template.
\end{IEEEkeywords}}


% make the title area
\maketitle


% To allow for easy dual compilation without having to reenter the
% abstract/keywords data, the \IEEEtitleabstractindextext text will
% not be used in maketitle, but will appear (i.e., to be "transported")
% here as \IEEEdisplaynontitleabstractindextext when the compsoc 
% or transmag modes are not selected <OR> if conference mode is selected 
% - because all conference papers position the abstract like regular
% papers do.
\IEEEdisplaynontitleabstractindextext
% \IEEEdisplaynontitleabstractindextext has no effect when using
% compsoc or transmag under a non-conference mode.



% For peer review papers, you can put extra information on the cover
% page as needed:
% \ifCLASSOPTIONpeerreview
% \begin{center} \bfseries EDICS Category: 3-BBND \end{center}
% \fi
%
% For peerreview papers, this IEEEtran command inserts a page break and
% creates the second title. It will be ignored for other modes.
\IEEEpeerreviewmaketitle


\section{Introduction}%
\label{sec:introduction}

\iffalse
######################
### 第一段写作材料 ###
######################
HostOS在虚拟化之前还需要实现大量的kernel subsystem(components) 如processing, memory, storage, and network capacity 更不要说要引入大量的LKM 这无疑极大的增加了hypervisor被危害的可能性。
All hypervisors need some operating system-level components—such as a memory manager, process scheduler, input/output (I/O) stack, device drivers, security manager, a network stack, and more—to run VMs.
It consists of a loadable kernel module, kvm.ko, that provides the core virtualization infrastructure and a processor specific module, kvm-intel.ko or kvm-amd.ko.

The linux kernel provides not only provides the core virtualization infrastructure but also some operating system components - such as memory manager, process scheduler, device drivers, security manager, and more.

% 在KVM虚拟化环境中,HostOS不仅要
在XEN环境中,hostos是啥,其主要就是下面的hypervisor,是否包括demo0?应该是不包括的。所以,在XEN环境中,HOSTOS的功能很简单,并不需要很多其他的功能,因此,我们的论文应该限定在KVM的环境中。
要不要提QMEU?不需要,因为KVM可以跟上面很多的软件构成虚拟化环境,但是无论如何都存在一个hostos,这个是由于KVM是一个内核模块的形式存在的原因,导致其还有很多其他的功能和模块。所以,这里只需要提KVM。

在KVM-based的vir环境中,hostos不仅含有核心的虚拟化组件(KVM)还含有大量的其他的kernel功能,如驱动设备 安全管理等。大量与虚拟化无关的组件,特别是drivers,提供了大量可被exploit的脆弱点。因为KVM是以内核模块存在的,所以,任何一个成功的对内核的exploit特别是提权攻击会导致整个虚拟化环境被攻陷。
 By adding virtualization capabilities to a standard Linux kernel, the virtualized environment can benefit from all the ongoing work on the Linux kernel itself.

#####################
### 第一段材料end ###
#####################

\fi

Integrating the hypervisor capabilities into a standard Linux kernel, the KVM-based virtualized environment not only benefits from all the ongoing work on the Linux kernel itself but also shares countless vulnerabilities with it. A successful exploit, especially a privilege escalation exploit, against the Linux kernel will also subvert the entire virtualization environment. 
Recently, researchers have proposed a lot of schemes to mitigate such a security concern. 
% A lot of works have been proposed to mitigate such a security concern.

% 将hypervisor功能整合到内核中,KVM虚拟化环境在获取了与内核同步的发展的优势的同时也不可避免的与内核共同面临相同的安全威胁。一个成功的针对内核的攻击,特别是提权攻击,同样会使得整个虚拟化环境被攻陷。研究者已经提出了大量的方案来解决整个安全问题。

% \paragraph{Hardware-assisted Schemes}%
% \label{par:hardware}
Hardware facilities, such as Memory Protection eXtensions(MPX)\cite{ramakesavan2015intel}, Encryption Instructions(AES-NI)\cite{gueron2010intel} , Software Guard eXtensions(SGX)\cite{mckeen2016intel}, and Secure Memory Encryption(SME)\cite{kaplan2016amd}, provide efficient memory isolation. These hardware facilities have already been actively used by researchers to separate the HostOS kernel components and the hypervisor component. 
% However, these schemes are limited to separate just a small portion of the hypervisor component.
% However, the strong memory isolation makes the access to the partition application to be strictly restricted.
However, most of these hardware facilities restrict access to the partition application to a narrow interface.
% However, access to the partition application in the protected memory has been strictly restricted with limited interfaces.
This makes it extremely tough to leverage these hardware facilities to separate the entire virtualization component from the kernel. Furthermore, most of these hardware facilities need developers to reconstruct their protected applications or build the whole program from scratch. 
% \paragraph{Reconstruct Hypervisor}%
% \label{par:reconstruct_hypervisor}
There are also some schemes that propose to separate the virtualization component from the rest of the kernel by reconstructing the HostOS or Hypervisor.
Microhypervisor advocates that the hypervisor should only be responsible for core virtualization privileges to reduce its interaction with the kernel. For example, Trustvisor and Nova adopt microhypervisor to achieve virtualization privilege separation and memory protection. 
Nested virtualization is another representative approach. 
Nested virtualization, such as CloudAuditor \cite{Wang2016CloudAuditorAC}, Nosv \cite{REN2017137}, CloudViosr-D \cite{mi2020mostly}, and the Turtles Project \cite{183330}, introduces a higher privilege level and isolated execution environment beyond the original hypervisor. 
Thus, separated privileges in the nested virtualization environment can no longer be subverted by the original one. For example, CloudVisor \cite{Zhang2011CloudVisor} proposed to use nested virtualization to decouple resource management components into a nested hypervisor. 
% Nevertheless, These schemes have been deemed unrealistic for the cloud providers in that cloud providers could not reconstruct their cloud architecture, let alone the significant performance losses introduced by nested virtualization.
% Nevertheless, These schemes have been deemed unrealistic for the cloud providers in that cloud providers could not reconstruct their cloud architecture, let alone the significant performance losses introduced by nested virtualization.
% Nevertheless, These schemes have been deemed unrealistic for the cloud providers in that cloud providers could not endure the significant performance losses introduced by nested virtualization, let alone reconstructing their cloud environment architecture.
Nevertheless, These schemes have been deemed incompatible for commercial cloud providers in that cloud providers could not endure the significant performance losses introduced by nested virtualization, let alone reconstructing their cloud environment architecture.

% reconstruct their cloud architecture, let alone the significant performance losses introduced by nested virtualization.

\iffalse
################################
###### 关于以往方案的描述 ######
################################
但是这些功能往往提供的是一个相对封闭的运行环境,其与外界的交互是十分受限的。
对在被这些硬件包括的区域中的代码和数据的访问是十分受限的,
并不能满足将整个虚拟化环境相关特权分离的环境要求。此外,这些硬件特性需要研发者重新设计他们的软件架构甚至是逻辑。
% apply different memory protection

% Thus, researchers separate some privileges into the nested hypervisor.
% . Nested virtualization
% These schemes
% reconstruct the Hypervisor
% 还有一些方案通过重构hypervisor来modularizing核心功能。一些方案提出使用microhypervisor的方式。另外一些方案则通过嵌套虚拟化引入更高的特权,将部分hyper的特权分离到一个更高的执行空间中。
% 嵌套虚拟化是另外一个具有代表性的方案,这些方案利用虚拟化会将特权分离的特点,将原本内核负责的虚拟化权限分离。
################################
###### 关于以往方案的描述 ######
################################

\fi

\iffalse
In this paper, we introduce a novel scheme, named HyperPS, to implement hypervisor monitoring based on privilege separation without relying on any special hardware or reconstructing the hypervisor. The key idea of HyperPS is to separate the core virtualization privileges, especially the privileges of managing the physical memory, into an isolated execution environment alongside the HostOS kernel,
so that the security of virtual machines can be strongly protected, even if the HostOS/hypervisor has been compromised. Our isolated execution does not rely on a special hardware feature, nor demand a higher privilege level. Our isolated execution actually shares the same privilege level with the HostOS kernel. 
\fi
Extended Page-Table Technology (EPT) is introduced by Intel to support the virtualization of physical memory.
% The Extended Page-Table (EPT) is used to support the virtualization of physical memory.
When EPT is in use, guest-physical addresses are translated by traversing a set of EPT paging structures to produce physical addresses that are used to access memory. Everything in the guest virtual machine`s memory cannot be safe if the HostOS/Hypervisor has been compromised. For example, a compromised HostOS/Hypervisor could remap one guest physical page frame to another physical page frame that holds the malicious code.
Extended Page Table Pointer (EPTP) is a hardware register that contains the address of the base of EPT table, as well as EPT configuration information. 
Virtual Machine Control Structure (VMCS) is the most important structure in a virtualization environment. It manages transitions into and out of virtual-machine extensions (VMX) non-root operations (VM entries and VM exits) as well as processor behavior in VMX non-root operation. 
The value of EPTP is configured and managed by VMCS. 
If the HostOS/Hypervisor has been subverted, the value of EPTP will no longer be safe. The attacker can load a new EPTP, thereby establishing a dedicated and malicious EPT Paging-structure hierarchy. 

\iffalse
any execution of the guest virtual machine will no longer be safe. 
For example, the adversary can redirect the execution of guest virtual machine to a unpredictable position by writing the \verb|rip| in guest-state area field in VMCS.

EPTP switching is one of these VM functions, which allows software (in both kernel and user mode) in guest VM to directly load a new EPTP, thereby establishing a different EPT Paging-structure hierarchy. 

Besides, the VMCS records the value of EPTP in the VM-Execution Control Fields. The processor will load the value in that field into the hardware register when VM-Entry. 
Even if the attacker cannot directly tamper the EPT, he can also subvert the guest VMs by tampering with a malicious EPTP value in the VMCS. Thus, HyperPS removed all VMCSes from the hypervisor too. 

Virtual Machine Control Structure (VMCS) is the most important structure in virtualization environment. It manages transitions into and out of virtual-machine extensions (VMX) non-root operations (VM entries and VM exits) as well as processor behavior in VMX non-root operation. 

In traditional virtualization architecture, the VMCS can only be accessed and managed by the HostOS/Hypervisor.
If the HostOS/Hypervisor has been subverted, any execution of the guest virtual machine will no longer be safe. For example, the adversary can redirect the execution of guest virtual machine to a unpredictable position by writing the \verb|rip| in guest-state area field in VMCS.
Extended Page-Table Technology (EPT) is introduced by Intel to support the virtualization of physical memory.
% The Extended Page-Table (EPT) is used to support the virtualization of physical memory.
When EPT is in use, guest-physical addresses are translated by traversing a set of EPT paging structures to produce physical addresses that are used to access memory. Everything in the guest virtual machine`s memory cannot be safe if the HostOS/Hypervisor has been compromised. For example, a compromised HostOS/Hypervisor could remap one guest physical page frame to another physical page frame that holds the malicious code.
\fi

In this paper, we introduce a novel scheme, named HyperPS, to implement hypervisor monitoring based on privilege separation without relying on any special hardware or reconstructing the hypervisor. The key idea of HyperPS is to separate the core virtualization privileges, especially the privileges of managing the physical memory, into an isolated execution environment alongside the HostOS kernel,
so that the security of virtual machines can be strongly protected, even if the HostOS/hypervisor has been compromised. Our isolated execution does not rely on a special hardware feature, nor demand a higher privilege level. Our isolated execution actually shares the same privilege level with the HostOS kernel. 


% In this paper, HyperPS separates the privilege of managing these two structures: VMCS (especially the EPTP filed in this structure) and EPT, from the HostOS/Hypervisor into an isolated execution environment. Any access during VM exit (any VMX root operation) to these two structures will be hooked and be redirected to the isolated execution environment. Code in the original HostOS/Hypervisor environment can no longer read or write VMCSs and EPTs directly.


% in VMX root operation

\iffalse
##############################
##### 关于我们工作的材料 #####
##############################
Virtual Machine Control Structure (VMCS) is the most important core structure in virtualization environment which is completely under the charge of the HostOS/Hypervisor. The VMCS manages transitions into and out of virtual-machine extensions (VMX) non-root operations (VM entries and VM exits) as well as processor behavior in VMX non-root operation. If the HostOS/Hypervisor has been subverted, the execution environment of guest VM will no longer be 
A compromised HostOS/Hypervisor can 
All management operations to the guest-VM are recorded in this structure.
The Extended Page-Table (EPT) is used to support the virtualization of physical memory. When EPT is in use, guest-physical addresses are translated by traversing a set of EPT paging structures to produce physical address that are used to access memory.
如果vmcs被篡改了,会对vm产生什么影响。
在一个虚拟化环境中,VMCS是干啥的,EPT是干啥的,讲出来,any virtualization operation 都将映射到这两个数据结构上,因此,我们将hostos对这两个结构体操作的权限从原本的内核中剥离
% the compromised HostOS/Hypervisor cannot subvert the virtual machines
从而使得已经被危害的hostOS无法通过虚拟化技术危害虚拟机。

上面讲了要做的是什么即权限分离,那么下面要写为什么要将VMCS EPT作为权限分离的根本,为什么这两个要放到隔离空间中去
% ,下面就要写为什么
, and replace these privileges into a dedicated space which can not be access by the HostOS anymore. 

##############################
##### 关于我们工作的材料 #####
##############################

\fi


We have implemented a full funtional prototype based on the KVM in Intel x86\_64 architecture.
In our prototype, we separate the privilege of managing these two structures: VMCS (especially the EPTP filed in this structure) and EPT, from the HostOS/Hypervisor into an isolated execution environment. Any access during VM exit (any VMX root operation) to these two structures will be hooked and be redirected to the isolated execution environment. Code in the original HostOS/Hypervisor environment can no longer read or write VMCSs and EPTs directly. 
Our prototype modified 300 SLOC (Source Lines of Code) of the original Linux kernel and introduced approximately 4K SLOC of new kernel code. 
We also conducted several security and performance experiments. The results show that HyperPS has gained an acceptable trade-off between security and performance.
% The expermental results show a trivial performance overhead.

% We have implemented a full funtional prototype based on the KVM in Intel x86\_64 architecture. Our prototype modified 300 SLOC (Source Lines of Code) of the original Linux kernel and introduced about 4K SLOC code. The expermental results show a trivial performance overhead.


\iffalse
在这里,并没有点出同层隔离的特点,我们并不只是在VMM被危害的情况下保护虚拟机,因为对应的方法有很多,我们的方法是同层隔离,这个也是上面的方案的缺点,已经说了方案的缺点,那么下面就是要说同层隔离这个东西,因为是我们的一个亮点。

\fi























\section{Background}%
\label{sec:background}

\subsection{QEMU-KVM}%
\label{sub:qemu_kvm}
QEMU is a generic and open source machine emulator and virtualizer. QEMU can use other hypervisor like \verb|Xen| or KVM to use CPU extensions for virtualization. When used as a virtualizer, QEMU achieves near native performances by executing the guest code directly on the host CPU.
Kernel-based Virtual Machine (KVM) is an open source virtualization technology that converts Linux into a type-1 (bare-metal) hypervisor. KVM is a part of the Linux kernel that shares all the linux kernel's operating system-level componets -such as the memory manager, process scheduler, security manager, and more to run VMs. Every VM is implemented as a regular Linux process, shceduled by the standard Linux scheduler, with dedicated virtual hardware like a network card, memory, and disks. KVM mainly consists of a loadable kernel module, kvm.ko, that provides the core virtualization infrastructure and a processor specific module, kvm-intel.ko or kvm-amd.ko.
In virtualization environment, KVM does not work on its own. It is only an API provided by the kernel for userspace. End users typically use KVM throgh QEMU where it is present as an acceleration method.
For the QEMU-KVM architecture, KVM interacts with QEMU (QEMU runs at the user space) in two ways: through device file \verb|/dev/kvm| and through memory mapped pages
% KVM interacts with user space - in most case QEMU - in two ways: through device file \verb|/dev/kvm| and through memory mapped pages.
Memory mapped pages are used for bulk transfer of data between QEMU and KVM. \verb|/dev/kvm| is the main API exposed by KVM. It supports a set of \verb|ioctl|s which allow QEMU to manage VMs and interact with them.
% 怎么引出HostOS的问题
% Since KVM is a part of the linux kernel, the Linux and the QEMU
% leverages the linux kernel's system-level



\subsection{VMCS}%
\label{sub:vmcs}
Virtual-machine Control Structure is a data structure used in Virtual Machine eXtensions (VMX). It controls VMX non-root operations (guest virtual machine execution operations) and VMX transitions. Access to the VMCS is managed through the VMCS pointer (one per logical processor). The VMCS pointer is read and written suing the instructions \verb|VMPTRST| and \verb|VMPTRLD|. In general, the hypervisor configures a VMCS using \verb|VMREAD|, \verb|VMWRITE|, and \verb|VMCLEAR| instructions. A hypervisor could use a different VMCS for each virtual machine that it supports. For a virtual machine with multiple logical processor, the hypervisor could use a different for each logical processor. A logical processor may also maintain a number of VMCSs that are active, however, at any given time, at most one of the active VMCSs is the \textbf{current} VMCS. VMX instructions operate only on the \textbf{current} VMCS. 

A compromised HostOS/hypervisor can force the guest virtual machine exit by tramper VM-Execution Control fields and VM-exit control fields in VMCS and tramper the guest virtual machine by writing Guest-State Area fields. 

% 一个被危害的hypervisor可以强迫虚拟机退出,并通过篡改其vmcs中的field等信息实现对guest VM 的危害。

% \subsection{Second Level Address Translation}%
% \label{sub:second_level_address_translation}
% Second Level Address Translation (SLAT) is a hardware-assisted virtualization technology which makes it possible to quick manage the physical memory without lots of VM exits. Extended Page Table (EPT) is the intel's implementation of SLAT, while ARM names its implementation as Stage-2 Page-tables.


\subsection{EPT}%
\label{sub:ept}
Intel implements it Second Level Address Translation (SLAT) as Extended Page Table (EPT). 
EPT allows each virtual machine to manage its page table (not the EPT), without giving access to the underlying host machine's MMU Hardware. Thus, EPT reduces the need for hypervisor to keep syncing the shadow pages eliminating the overhead.
If EPT is enabled, guest-physical addresses are translated by traversing a set of EPT paging structures to produce physical addresses that are used to access memory.
In addition to translating a guest-physical address to a physical address, EPT specifies the privileges that software is allowed when accessing the address. Attempts at disallowed accesses are called EPT violations and cause VM exits.

A compromised HostOS/hypervisor could tramper EPT paging structures so that the virtual machine will execute arbitrary malicious code without knowing it. 
Moreover, a compromised HostOS/Hypervisor could access any data in the guest virtual machine with the help of virtual machine introspection.

% the mapping relationship





















% \IEEEtriggercmdraisesectionheading{\section{Introduction}\label{sec:introduction}}
% Computer Society journal (but not conference!) papers do something unusual
% with the very first section heading (almost always called "Introduction").
% They place it ABOVE the main text! IEEEtran.cls does not automatically do
% this for you, but you can achieve this effect with the provided
% \IEEEraisesectionheading{} command. Note the need to keep any \label that
% is to refer to the section immediately after \section in the above as
% \IEEEraisesectionheading puts \section within a raised box.




% The very first letter is a 2 line initial drop letter followed
% by the rest of the first word in caps (small caps for compsoc).
% 
% form to use if the first word consists of a single letter:
% \IEEEPARstart{A}{demo} file is ....
% 
% form to use if you need the single drop letter followed by
% normal text (unknown if ever used by the IEEE):
% \IEEEPARstart{A}{}demo file is ....
% 
% Some journals put the first two words in caps:
% \IEEEPARstart{T}{his demo} file is ....
% 
% Here we have the typical use of a "T" for an initial drop letter
% and "HIS" in caps to complete the first word.
% \IEEEPARstart{T}{his} demo file is intended to serve as a ``starter file''
% for IEEE Computer Society journal papers produced under \LaTeX\ using
% IEEEtran.cls version 1.8b and later.
% % You must have at least 2 lines in the paragraph with the drop letter
% % (should never be an issue)
% I wish you the best of success.

% \hfill mds
 
% \hfill August 26, 2015

% \subsection{Subsection Heading Here}
% Subsection text here.

% needed in second column of first page if using \IEEEpubid
%\IEEEpubidadjcol

% \subsubsection{Subsubsection Heading Here}
% Subsubsection text here.


% An example of a floating figure using the graphicx package.
% Note that \label must occur AFTER (or within) \caption.
% For figures, \caption should occur after the \includegraphics.
% Note that IEEEtran v1.7 and later has special internal code that
% is designed to preserve the operation of \label within \caption
% even when the captionsoff option is in effect. However, because
% of issues like this, it may be the safest practice to put all your
% \label just after \caption rather than within \caption{}.
%
% Reminder: the "draftcls" or "draftclsnofoot", not "draft", class
% option should be used if it is desired that the figures are to be
% displayed while in draft mode.
%
%\begin{figure}[!t]
%\centering
%\includegraphics[width=2.5in]{myfigure}
% where an .eps filename suffix will be assumed under latex, 
% and a .pdf suffix will be assumed for pdflatex; or what has been declared
% via \DeclareGraphicsExtensions.
%\caption{Simulation results for the network.}
%\label{fig_sim}
%\end{figure}

% Note that the IEEE typically puts floats only at the top, even when this
% results in a large percentage of a column being occupied by floats.
% However, the Computer Society has been known to put floats at the bottom.


% An example of a double column floating figure using two subfigures.
% (The subfig.sty package must be loaded for this to work.)
% The subfigure \label commands are set within each subfloat command,
% and the \label for the overall figure must come after \caption.
% \hfil is used as a separator to get equal spacing.
% Watch out that the combined width of all the subfigures on a 
% line do not exceed the text width or a line break will occur.
%
%\begin{figure*}[!t]
%\centering
%\subfloat[Case I]{\includegraphics[width=2.5in]{box}%
%\label{fig_first_case}}
%\hfil
%\subfloat[Case II]{\includegraphics[width=2.5in]{box}%
%\label{fig_second_case}}
%\caption{Simulation results for the network.}
%\label{fig_sim}
%\end{figure*}
%
% Note that often IEEE papers with subfigures do not employ subfigure
% captions (using the optional argument to \subfloat[]), but instead will
% reference/describe all of them (a), (b), etc., within the main caption.
% Be aware that for subfig.sty to generate the (a), (b), etc., subfigure
% labels, the optional argument to \subfloat must be present. If a
% subcaption is not desired, just leave its contents blank,
% e.g., \subfloat[].


% An example of a floating table. Note that, for IEEE style tables, the
% \caption command should come BEFORE the table and, given that table
% captions serve much like titles, are usually capitalized except for words
% such as a, an, and, as, at, but, by, for, in, nor, of, on, or, the, to
% and up, which are usually not capitalized unless they are the first or
% last word of the caption. Table text will default to \footnotesize as
% the IEEE normally uses this smaller font for tables.
% The \label must come after \caption as always.
%
%\begin{table}[!t]
%% increase table row spacing, adjust to taste
%\renewcommand{\arraystretch}{1.3}
% if using array.sty, it might be a good idea to tweak the value of
% \extrarowheight as needed to properly center the text within the cells
%\caption{An Example of a Table}
%\label{table_example}
%\centering
%% Some packages, such as MDW tools, offer better commands for making tables
%% than the plain LaTeX2e tabular which is used here.
%\begin{tabular}{|c||c|}
%\hline
%One & Two\\
%\hline
%Three & Four\\
%\hline
%\end{tabular}
%\end{table}


% Note that the IEEE does not put floats in the very first column
% - or typically anywhere on the first page for that matter. Also,
% in-text middle ("here") positioning is typically not used, but it
% is allowed and encouraged for Computer Society conferences (but
% not Computer Society journals). Most IEEE journals/conferences use
% top floats exclusively. 
% Note that, LaTeX2e, unlike IEEE journals/conferences, places
% footnotes above bottom floats. This can be corrected via the
% \fnbelowfloat command of the stfloats package.




% \section{Conclusion}
% The conclusion goes here.





% if have a single appendix:
%\appendix[Proof of the Zonklar Equations]
% or
%\appendix  % for no appendix heading
% do not use \section anymore after \appendix, only \section*
% is possibly needed

% use appendices with more than one appendix
% then use \section to start each appendix
% you must declare a \section before using any
% \subsection or using \label (\appendices by itself
% starts a section numbered zero.)
%

%
% \appendices
% \section{Proof of the First Zonklar Equation}
% Appendix one text goes here.
%
% % you can choose not to have a title for an appendix
% % if you want by leaving the argument blank
% \section{}
% Appendix two text goes here.
%

% use section* for acknowledgment
\ifCLASSOPTIONcompsoc
  % The Computer Society usually uses the plural form
  \section*{Acknowledgments}
\else
  % regular IEEE prefers the singular form
  \section*{Acknowledgment}
\fi


The authors would like to thank...


% Can use something like this to put references on a page
% by themselves when using endfloat and the captionsoff option.
\ifCLASSOPTIONcaptionsoff
  \newpage
\fi



% trigger a \newpage just before the given reference
% number - used to balance the columns on the last page
% adjust value as needed - may need to be readjusted if
% the document is modified later
%\IEEEtriggeratref{8}
% The "triggered" command can be changed if desired:
%\IEEEtriggercmd{\enlargethispage{-5in}}

% references section

% can use a bibliography generated by BibTeX as a .bbl file
% BibTeX documentation can be easily obtained at:
% http://mirror.ctan.org/biblio/bibtex/contrib/doc/
% The IEEEtran BibTeX style support page is at:
% http://www.michaelshell.org/tex/ieeetran/bibtex/
%\bibliographystyle{IEEEtran}
% argument is your BibTeX string definitions and bibliography database(s)
%\bibliography{IEEEabrv,../bib/paper}
%
% <OR> manually copy in the resultant .bbl file
% set second argument of \begin to the number of references
% (used to reserve space for the reference number labels box)
\begin{thebibliography}{1}

\bibitem{IEEEhowto:kopka}
H.~Kopka and P.~W. Daly, \emph{A Guide to \LaTeX}, 3rd~ed.\hskip 1em plus
  0.5em minus 0.4em\relax Harlow, England: Addison-Wesley, 1999.

\end{thebibliography}

% biography section
% 
% If you have an EPS/PDF photo (graphicx package needed) extra braces are
% needed around the contents of the optional argument to biography to prevent
% the LaTeX parser from getting confused when it sees the complicated
% \includegraphics command within an optional argument. (You could create
% your own custom macro containing the \includegraphics command to make things
% simpler here.)
%\begin{IEEEbiography}[{\includegraphics[width=1in,height=1.25in,clip,keepaspectratio]{mshell}}]{Michael Shell}
% or if you just want to reserve a space for a photo:

\begin{IEEEbiography}{Michael Shell}
Biography text here.
\end{IEEEbiography}

% if you will not have a photo at all:
\begin{IEEEbiographynophoto}{John Doe}
Biography text here.
\end{IEEEbiographynophoto}

% insert where needed to balance the two columns on the last page with
% biographies
%\newpage

\begin{IEEEbiographynophoto}{Jane Doe}
Biography text here.
\end{IEEEbiographynophoto}

% You can push biographies down or up by placing
% a \vfill before or after them. The appropriate
% use of \vfill depends on what kind of text is
% on the last page and whether or not the columns
% are being equalized.

%\vfill

% Can be used to pull up biographies so that the bottom of the last one
% is flush with the other column.
%\enlargethispage{-5in}



% that's all folks
\end{document}


